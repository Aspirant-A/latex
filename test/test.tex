\documentclass{ctexart}
\usepackage{amsfonts}
% \renewcommand{\baselinestretch}{1.25}
% \ctexset{
%     section={format={\heiti \zihao{4}}},
%     subsection={format={\heiti \zihao{6} \bfseries}, beforeskip=0pt, afterskip=0pt },
%     subsubsection={format={\kaishu \zihao{5}}, beforeskip=0pt, afterskip=0pt},
% }

% \usepackage[colorlinks, urlcolor=black, linkcolor=black]{hyperref}

% \usepackage[left=2.5cm,right=1.97cm,top=2.5cm,bottom=2.5cm]{geometry}

% \usepackage{fontspec}
% \setmainfont{Times New Roman}
% \usepackage{ctex}

% \documentclass{book}
% \usepackage{amsmath}


% \title{\heiti \zihao{2}用 \LaTeX 排版一篇中文期刊论文}
% \author{\kaishu \zihao{-4}赵竞昂}
% \date{\today}


\begin{document}
ojaidsfhoa
% \maketitle
% \tableofcontents
% \noindent \heiti 摘要:\songti 本文主要针对\LaTeX 
为中文论文排版进行简要的描述,针对一些常见的问题给出具体的解决办法,
例如:设置各级标题格式、插入脚注、使用国标格式插入参考文献。
文章主要以工业工程该期刊的要求作为模板,通过十次课程进行讲解。

\noindent\heiti 关键词:\songti \LaTeX ; 中文期刊 ;工业工程 \par
\noindent\heiti 中图分类号:\songti 000 \hspace{3cm}     \heiti 文献标识码:\songti A  \\
%英文标题居中加粗
% \begin{center}
% \textbf{\zihao{4} How to Write An Chinese Journal with \LaTeX} \\ 
% \zihao{5} Joefsong \\ 
% \zihao{-5} (Heiheihei College Papapa School, QQ Group 970479548)\\
% \end{center}
% \textbf{Abstract:} Wo de ying yu hen lan, jiu bu fan yi le. Fan zheng wo jue de pai ban he fan yi ye mei you shen me guan xi. Dan shi wei le rang ying wen zhai yao nei rong xian de feng man hai shi yao duo xie yi dian. hao le wo bian bu xia qu le. Jiu zhe yang ba.\\
% \textbf{key words:}  \LaTeX ; Chinese Journal; Industrial Engineering Journal
% \section{页面格式设置}%一级标题
	期刊论文涉及到页面的设置主要有页边距和页眉页脚。页边距的设置可以利用geometry宏包,对上下左右的边距进行设置,而针对于页眉页脚,本节仅简要介绍fancy的基本用法。
% \section{基本文本格式以及段落调整}%一级标题
期刊论文的要求中,要求最多的就是字体字号以及章节格式,因此本节简要介绍\LaTeX 中的中英文字体字号设置以及段落格式的基本要求的实现。
\subsection{字体字号}%二级标题
\LaTeX 中若使用了ctex或者ctexart宏包,则需要注意\LaTeX 包含常用的四种中文字体以及三种英文字体可供选择。由于期刊论文一般不要求其他字体,因此本节不在讲述载入系统字体的方法。
\subsubsection{字体}%三级标题
\LaTeX 默认的中文字体为宋体, 且有黑体、 楷书和仿宋三种气体可供选择; 默认的英文字体为罗马字体 (Roman Family) ,且有无衬线字体 (Sans Serif Family) 以及打印字体 (Typewtiter Family) 可供选择。具体字体及其命令如下:

中文字体	命令	英文字体	命令

宋体	songti	罗马字体	rmfamily

黑体	heiti	无衬线字体	sffamily

楷体	kaishu	打印字体	ttfamily

仿宋	fangsong		

\subsubsection{字号}
由于使用了中文的宏包,因此在设置字号的时候可以直接使用 zihao 命令进行定义字号大小。大括号中是字号的大小,如五号字体为5,小五号字体为-5。然而当需要在导言区当定义一些字体字号格式的时候,如图表题注的字号字体的时候,需要利用最基本的定义字体字号的方式,此处笔者还未找到其他的解决方法。基本字号设置命令以及其对应的字号如下:

字号命令	文档默认字号			中文默认字体\\
10pt	11pt	12pt	c5size \\
Tiny	5	6	6	七\\
Scriptsize	7	8	8	小六\\
Footnotesize	8	9	10	六\\
small 	9	10	10.95	小五\\
normalsize 	10	10.95	12	五\\
large	12	12	14.4	小四\\
Large	14.4	14.4	17.28	小三\\
LARGE 	17.28	17.28	20.74	小二\\
Huge	20.74	20.74	24.88	二 \\
Huge 	24.88	24.88	24.88	一\\
% \section{段落调整}

\subsection{基本段落格式}
需要注意的是\LaTeX 在编辑中文文本的时候会忽略空格和回车,两个中文字符之间不管有多少空格,生成的PDF文件都不会显示出来,因此如果想要实现空格,需要利用空格命令。此外单个回车也不会是生成全新段落,若要实现生成新的段落需要在段落间空一行。若仅需要换行而不需要生成新的段落可以用双反斜杠实现。利用空行生成的新段落会默认首行缩进两个字符,利用双反斜杠生成的新段落不会首行缩进。若想要取消某段落首行缩进,须在段首使用 noident 命令。而中文文本和英文文本在一起出现的 时候,会自动在两者之间生成空格。
quad 一个空格  qquad 两个空格  hspace 指定长度空格   hfill 两个字符间填充成空格

\subsection{行间距调整}
\LaTeX 使用中文ctex 宏包之后,默认的是1.3倍行距,而中文期刊不会对行距有特别的要求,因此可直接使用默认行距,若需要设置论文行距,可以在导言区重新定义baselinestretch的值,具体参见导言区命令。各级标题与段落之间的间距设置将在\ref{多级标题}%讲多级标题的章节
给出。

也会遇到情况如摘要和作者的间距太长,仅需提高摘要的位置,使其靠近作者或者标题,此实仅需利用vspace命令定义向上或向下移动的距离。常见距离单位如下:也可通过baselineskip 或 textwidth 等定义相对距离。
单位 	名称 	说明
mm 	毫米 	1 mm = 2.845 pt
pt 	点 	1 pt = 0.351 mm
cm 	厘米 	1 cm= 10 mm= 28.453 pt
in 	英寸 	1 in = 25.4 mm = 72.27 pt
ex 	ex 	1 ex = 当前字体尺寸中 x 的高度
em 	em 	1 em = 当前字体尺寸中 M 的宽度

\subsection{自动编号} \label{自动编号}
若在论文写作过程中有分条的需求可利用itemize或者enumerate 环境实现分条和交叉引用。itemize 可以满足自定义的序号标识,而enumerate则可以自动生成有序的标号。类似的\LaTeX 中也包含有定理定义证明等环境可以自动生成序号以及可以实现交叉引用,但是由于在期刊论文中的实用性不高,因此本文不予详细介绍。

\section{多级标题以及文件导入} \label{多级标题}
\LaTeX 在编辑过程中,通过命令可以自动生成有序的标题,并可以在导言区对其格式进行设置。然而在写文章的过程中,没有人可以一气呵成,然而由于在利用\LaTeX 写作过程中,会出现很多命令,并且当篇幅过长的时候影响审阅,以及会使运行时间变得越来越长,因此将不同章节份文件保存,最后进行整合可以使文章看起来简介有序,并且在无需更改的时候可以先行注释不予运行,加快运行速度,提高编辑效率。 

\subsection{多级标题格式设置}
\LaTeX 默认有三级标题,而在实际论文的写作过程中很少用到四级标题,因为会使标题序号变得异常的繁琐且不美观,如果需要四级标题,可直接利用\ref{自动编号}%讲自动编号的章节
中的itemize 环境实现。

而三级标题在未使用ctex中文宏包之前,默认为左对齐。在使用ctex宏包之后依然为左对齐,但是在使用ctexart宏包之后会默认为居中对齐。由于本文使用的文件类型为ctexart因此在需要使一级标题左对齐的时候需要在导言区进行设置。各级标题及其对应名命令如下:\\
一级标题 section\\
二级标题 subsection\\
三级标题 subsubsection

对各级标题格式进行设置可以利用ctexsetup语句,但是此语句并非实现此功能的唯一方法,其他方法不予详细介绍。本文主要针对工业工程期刊的要求,对各级标题进行字体字号设置,此外针对标题与段落间距的设置也会进行简要介绍,由于笔者暂未发现其他设置需求,因此本文不予介绍。
\subsection{文件导入}
实现章节以单独文件的形式保存,仅需将章节及文本内容单独一tex类型文件保存,且在正文中利用input 命令或者include 命令进行载入。 需要注意的是,当章节内容单独保存的时候不需要(不能)添加导言区内容;此外,利用input 命令载入的时候,文本会直接在段落后载入,而include 载入的时候则会在新的页面生成段落章节。因此可以根据自己需要选择使用,但期刊论文一般使用input命令即可。

% 附录:
% 	\begin{enumerate}
% 	\item 导入章节input 以及 include 的用法及区别:\\		
% 		\url{https://blog.csdn.net/weixin_42919606/article/details/82939495}	
% 	\item  标题格式更改\\		
% 		\url{https://www.jianshu.com/p/d7848f815e5f}		
% 	\item 编号方法itemize 和 enumerate :\\		
% 		\url{http://blog.sina.com.cn/s/blog_77f5a65c0101fmjl.html}\\		
% 		\url{https://blog.csdn.net/qq_18055167/article/details/83714725}		
% 	\item \label{字体字号} 字体字号及相关设置:\\		
% 		\url{https://blog.csdn.net/weixin_44537194/article/details/87720878}\\	
% 		\url{https://blog.csdn.net/ujsdui/article/details/79075327}
% 	\item 常用命令:\\		
% 		\url{https://blog.csdn.net/garfielder007/article/details/51646802}
% 	\item 文档类型以及页面设置的基本介绍:\\
% 		\url{https://blog.csdn.net/wei_love_2017/article/details/86617235}
% 	\end{enumerate}	
	
	



\end{document}

% \title{\LaTeX{} 学习手册}
% \author{赵竞昂}
% \maketitle

% \tableofcontents

% \section{第一节}
% \subsection{小结}
% \subsection{小结}
% \subsection{小结}
% \section{第二节}
% \subsection{小结}
% \subsection{小结}
% \subsection{小结}
% \section{第三节}
% \subsection{小结}
% \subsection{小结}
% \subsection{小结}
% \section{第四节}
% \subsection{小结}
% \subsection{小结}
% \subsection{小结}
% \section{第五节}
% \subsection{小结}
% \subsection{小结}
% \subsection{小结}
% \subsection{小结}
%段落缩进
% \parindent=0pt

%下划线
% \usepackage[normalem]{ulem} 
% \usepackage{xeCJKfntef}

%颜色
% \usepackage{xcolor}

% \begin{document}
% \begin{flushleft}
%     \hspace{1em} 在$\bigtriangleup OFP$,由三边关系,得$OP+OF>FP$,\\
%     由于$OP$和$OF$为定值,所以若$O,F,P$构成三角形,\\
%     那么一定会有$OP+OF>FP$,但是在旋转的过程\\
%     中会出现$O,F,P$共线,此时会有$OP+OF=FP$,所\\
%     以,此时$FP$会有最大值。
% \end{flushleft}
% \begin{flushleft}
%     \hspace{1em} 此时,大家一定能注意到,若$P$继续运动\\
%     一定会出现$O,F,P$再次共线,在这段运动过程中,\\
%     我能由三边关系,得$OF-OP<FP$,那我\\
%     想问什么时候$FP$会有最小值呢,此时大家肯定\\
%     会告诉我在下一次$O,F,P$共线时,因为此时会\\
%     有$OF-OP=FP$,所以,此时$FP$会有最小值。
% \end{flushleft}
% \begin{center}
%     \uline{圆外一点到圆周上一动点的距离:\\
%     {\color{blue}最大值为定点到圆心的距离加半径,}\\
%     最小值为定点到圆心的距离减半径。}
% \end{center}

% \begin{center}
%     若大家没看懂,那就让你们常老师教你们哟!!!
% \end{center}

% 在\LaTeX{}中排版中文。
% 汉字和English单词混排,通常不需要在中英文之间添加额外的空格。
% 当然,为了代码的可读性,加上汉字和 English 之间的空格也无妨。
% 汉字换行时不会引入多余的空格。

%换行命令\par 也可以用空行来代替
% Several spaces equal one.

% Front spaces are ignored.

% An empty line starts a new
% paragraph.\par
% A \verb|\par| command does
% the same.

% \end{document} 