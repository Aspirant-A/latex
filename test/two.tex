\section{基本文本格式以及段落调整}%一级标题
期刊论文的要求中,要求最多的就是字体字号以及章节格式,因此本节简要介绍\LaTeX 中的中英文字体字号设置以及段落格式的基本要求的实现。
\subsection{字体字号}%二级标题
\LaTeX 中若使用了ctex或者ctexart宏包,则需要注意\LaTeX 包含常用的四种中文字体以及三种英文字体可供选择。由于期刊论文一般不要求其他字体,因此本节不在讲述载入系统字体的方法。
\subsubsection{字体}%三级标题
\LaTeX 默认的中文字体为宋体, 且有黑体、 楷书和仿宋三种气体可供选择; 默认的英文字体为罗马字体 (Roman Family) ,且有无衬线字体 (Sans Serif Family) 以及打印字体 (Typewtiter Family) 可供选择。具体字体及其命令如下:

中文字体	命令	英文字体	命令

宋体	songti	罗马字体	rmfamily

黑体	heiti	无衬线字体	sffamily

楷体	kaishu	打印字体	ttfamily

仿宋	fangsong		

\subsubsection{字号}
由于使用了中文的宏包,因此在设置字号的时候可以直接使用 zihao 命令进行定义字号大小。大括号中是字号的大小,如五号字体为5,小五号字体为-5。然而当需要在导言区当定义一些字体字号格式的时候,如图表题注的字号字体的时候,需要利用最基本的定义字体字号的方式,此处笔者还未找到其他的解决方法。基本字号设置命令以及其对应的字号如下:

字号命令	文档默认字号			中文默认字体\\
10pt	11pt	12pt	c5size \\
Tiny	5	6	6	七\\
Scriptsize	7	8	8	小六\\
Footnotesize	8	9	10	六\\
small 	9	10	10.95	小五\\
normalsize 	10	10.95	12	五\\
large	12	12	14.4	小四\\
Large	14.4	14.4	17.28	小三\\
LARGE 	17.28	17.28	20.74	小二\\
Huge	20.74	20.74	24.88	二 \\
Huge 	24.88	24.88	24.88	一\\